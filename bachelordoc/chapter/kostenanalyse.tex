%----------------- KAPITEL : BEISPIELE  ----------------- %	
\chapter{Kostenanalyse}
\label{cha:kostenanalyse}

\chaptersummary{\sblindtext

Ja also darum gehts\dots}


\section{App Engine}
\blindtext

\subsection*{Skalierung}
\blindtext

\subsection*{Kosten}
\sblindtext


\section{SQL}
\blindtext

\subsection*{Skalierung}
\blindtext
Vertikale Skalierung, da:
 - Einfache Nutzerdaten
 - Datenstamm gering
Wichtig:
 - Zugriffsgeschwindigkeit, daher SQL



\subsection*{Kosten}
High availabilty wird nicht benötigt, da auf der Datenbank gespeichert wird, ob es sich bei einem Nutzer um einen Arzt handelt.
Falls ein User ein Arzt ist, so kann er seine eigene Praxis Seite erstellen.
Somit verliert die Applikation nur die Fuktionalität der Praxeneröffnung.
\sblindtext


\section{Cloud Run}
\blindtext

\subsection*{Skalierung}
\blindtext

\subsection*{Kosten}
\sblindtext


\section{Firestore}
\blindtext

\subsection*{Skalierung}
\blindtext

\subsection*{Kosten}
\sblindtext


\section{Firebase}
\blindtext

\subsection*{Skalierung}
\blindtext

\subsection*{Kosten}
\sblindtext


\section{Artifact Registry}
\blindtext
\subsection*{Kosten}
Bei der Artifact Registry entstehen keine Kosten, da das 
UserAPI Image kleiner als 0.5GB ist und somit die Google 
Cloud Kostenschwelle nicht überschreitet.
\sblindtext


\section{Container Registry}
\blindtext

\subsection*{Skalierung}
\blindtext

\subsection*{Kosten}
\sblindtext


\section{Cloud Function}
\blindtext

\subsection*{Skalierung}
\blindtext

\subsection*{Kosten}
\sblindtext


\section{Cloud Build}
\blindtext

\subsection*{Skalierung}
\blindtext

\subsection*{Kosten}
\sblindtext


\section{IAM}
\blindtext

\subsection*{Skalierung}
\blindtext

\subsection*{Kosten}
\sblindtext



\section{Zusammenfassung}
Hier die Kosten zusammenfassen